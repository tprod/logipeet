documentclass[a4paper, 12pt]{article}
usepackage[portuges]{babel}
usepackage[utf8]{inputenc}
usepackage{amsmath}
usepackage{mathtools}
usepackage{indentfirst}
usepackage[export]{adjustbox}
usepackage{graphicx,wrapfig}
usepackage{multicol}
usepackage{listings}
usepackage{xcolor}

definecolor{codegreen}{rgb}{0,0.6,0}
definecolor{codegray}{rgb}{0.5,0.5,0.5}
definecolor{codepurple}{rgb}{0.58,0,0.82}
definecolor{backcolour}{rgb}{0.95,0.95,0.92}

lstdefinestyle{mystyle}{
    backgroundcolor=color{backcolour},   
    commentstyle=color{codegreen},
    keywordstyle=color{magenta},
    numberstyle=tinycolor{codegray},
    stringstyle=color{codepurple},
    basicstyle=ttfamilyfootnotesize,
    breakatwhitespace=false,         
    breaklines=true,                 
    captionpos=b,                    
    keepspaces=true,                 
    numbers=left,                    
    numbersep=5pt,                  
    showspaces=false,                
    showstringspaces=false,
    showtabs=false,                  
    tabsize=2
}

lstset{style=mystyle}







begin{document}
%maketitle

begin{titlepage}
	begin{center}
	
		Huge{Universidade de Aveiro}
		large{Departamento de Electrónica, Telecomunicações e Informática} 
		vspace{15pt}
        vspace{200pt}
        textbf{LARGE{DigiPeet}}
        large{Base De Dados}
		%title{{large{Título}}}
		vspace{150pt}
	end{center}
	
	begin{flushleft}
		begin{tabbing}
			Professor Pedro Fonseca
			vspace{1cm}
			Alunos João Lima 48019 
			hspace{1,4cm}
			Rafael Fonseca 93355
			
	end{tabbing}
 end{flushleft}
	vspace{1cm}
	
	begin{center}
		vspace{fill}
			 Março
		 2021
			end{center}
end{titlepage}

% % % % % % % % % % % % % % % % % % % % % % % % % %

newpage
tableofcontents
thispagestyle{empty}

% % % % % % % % % % % % % % % % % % % % % % % % % %
newpage
pagenumbering{arabic}

% % % % % % % % % % % % % % % % % % % % % % % % % %

newpage
section{Introdução}
hspace{0.6cm}
Este relatório tem como objetivo demonstrar como foi implementadada a base de dados para aplicação DigiPeet. A aplicação é uma ferramenta para o utilizador, que tem como objetivo facilitar a sua experiência. Nela o ultizador poderá criar uma conta pessoal, fazer gestão do armazémstock, aceder a um calendário de atividades e de vacinas e configurar o Feed eet.

% % % % % % % % % % % % % % % % % % % % % % % % % %

section{Descrição Geral}
hspace{0.6cm}
A base de dados tem com função guardar e atualizar autonomamente todos os dados provenientes da aplicação e do dispensador, garantindo assim o seu bom funcionamnto.

% % % % % % % % % % % % % % % % % % % % % % % % % %

newpage
section{Tipo de Utilizadores}
hspace{0cm}
subsection{Administrador}
- Tem a função criar ou editar os utilizadores da aplicação;
hspace{0cm}
subsection{Voluntário}
- Utilizadores da aplicação.

% % % % % % % % % % % % % % % % % % % % % % % % % %

newpage
section{Descrição dos Use-Cases}
vspace{2.5cm}

subsection{Todos os Utilizadores}
vspace{0.5cm}

- Login O utilizador entra na aplicação, o sistema regista a hora de entrada e apresenta ao utilizador a sua área de trabalho (Cada tipo de utilizador tem um login diferente).

- Logout O utilizador sai da aplicação, o sistema regista a hora de saída, e é fechada a área de trabalho. 

vspace{15pt}

subsection{Administrador}
vspace{0.5cm}

- Aprovar Contas Aprovar o registo de novas contas na aplicação. Só depois deste passo é que as contas são adicionadas a base de dados.

- Editar Contas Permite activardesactivar as contas existentes, bem como alterar o nome e a password

- Lista das Contas Obter uma listagem com todas as contas de utilizadores. 

newpage
subsection{Voluntário}
vspace{1cm}

subsubsection{Registo na app}
vspace{0.5cm}

- Criar Conta Preenchimento de um formulário com nome completo, email, idade, género e tipo de utilizador. Também é necessária a escolha de um username e uma password para fazer o login na aplicação.

subsubsection{Menu}
vspace{0.5cm}

-Ler código de barras O volutário pode ler códigos de barras de produtos através da camera do smartphone, para isso o utilizador precisa aceitar as permissões de acesso à camera.

-Adicionar produtos Após a leitura do código de barras, caso esse produto não exista na aplicação, o utilizador pode adicioná-lo.  

-Aceder ao calendário O utilizador pode aceder ao calendário e ver as ativades futuras.

-Ver inventário O utilizador pode ver os produtos que existem em armazém.

-Atualizar stock O utilizador pode adicionarretirar produtos ao stock.

-Aceder ao dispensador O utilizador pode controlar o dispensador a partir da aplicação.

-Alterar definições O utilizador pode alterar as definições da aplicação.


newpage
section{Descrição do Diagrama de Classes}
vspace{2.5cm}

newpage
section{Conversão do Diagrama de Classes para o Modelo 
Relacional}
vspace{2.5cm}

newpage
section{Descrição do Modelo Físico}
vspace{2.5cm}
hspace{-1cm}includegraphics[width=15cm]{bd.png}

newpage
section{Criação de Tabelas}
vspace{2.5cm}
De acordo com o modelo físico é gerado o sguinte código SQL para criação das tabelas. Este processo é automatizado através do software Mysql Workbench.
Nestas tabelas são utilizadas chaves primárias e chaves estrangeiras de modo a podermos aceder às várias tabelas conforme o especificado no modelo físico.

vspace{1cm}
lstinputlisting[language=SQL]{logipeet2_db.sql}

newpage
section{Referências}
vspace{2.5cm}


end{document}